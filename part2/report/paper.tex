\documentclass{scrartcl}
\usepackage{amsmath}
\usepackage{fancyhdr}
\usepackage{minted}
\usepackage{enumitem}
\usepackage{hyperref}
\usepackage{multirow}
\usepackage[table]{xcolor}
\usepackage{svg}
\usepackage{caption}
\usepackage{subcaption}
\usepackage{scalerel}
\usepackage{amsfonts}

\author{Jelmer Firet (s1023433, \href{mailto:jelmer.firet@ru.nl}{jelmer.firet@ru.nl}) \and
Bram Pulles (s1015194, \href{mailto:bram.pulles@ru.nl}{bram.pulles@ru.nl})}
\title{\textbf{Automated Reasoning\\Practical Assignment -- Part 1}}

\pagestyle{fancy}
\fancyhf{}
\fancyhead[L]{Automated Reasoning PA 2|Jelmer Firet \& Bram Pulles}
\fancyhead[R]{\thepage}

\setminted{autogobble,tabsize=4,fontsize=\scriptsize}
\setlength{\parindent}{0pt}


\begin{document}
\maketitle

\section{Groups} % (fold)
\label{sec:groups}
\begin{enumerate}[label=\alph*)]
\item 
	Using prover9 we can prove that $I*x=x$, $inv(inv(x))=x$ and $inv(x)*x=I$.\\
	However, it is not true in general that $inv(x*y)=inv(x)*inv(y)$.\\
	The smallest group for which this does not hold has size $6$.
\item 
	The smallest non-Abelian group has size $6$.
\item 
	For $n=2$ all such groups are Abelian.\\
	For $n=3$ the group is not Abelian in general and the smallest counterexample has size $27$.\\
	For $n=4$ the group is not Abelian and the smallest counterexample has size $8$.
\end{enumerate}
% section groups (end)

\end{document}
