\documentclass{scrartcl}

\usepackage{amsmath}
\usepackage{fancyhdr}
\usepackage{minted}
\usepackage{enumitem}
\usepackage{hyperref}
\usepackage{multirow}
\usepackage[table]{xcolor}
\usepackage{svg}
\usepackage{caption}
\usepackage{subcaption}
\usepackage{scalerel}
\usepackage{amsfonts}

\author{Jelmer Firet (s1023433, \href{mailto:jelmer.firet@ru.nl}{jelmer.firet@ru.nl}) \and
Bram Pulles (s1015194, \href{mailto:bram.pulles@ru.nl}{bram.pulles@ru.nl})}
\title{\textbf{Automated Reasoning\\Practical Assignment -- Part 2}}

\pagestyle{fancy}
\fancyhf{}
\fancyhead[L]{Automated Reasoning PA 2|Jelmer Firet \& Bram Pulles}
\fancyhead[R]{\thepage}

\setminted{autogobble,tabsize=4,fontsize=\scriptsize}
\setlength{\parindent}{0pt}


\begin{document}
\maketitle

\section{Groups} % (fold)
\label{sec:groups}
\begin{enumerate}[label=\alph*)]
\item
	We proved that $I*x=x$, $inv(inv(x))=x$ and $inv(x)*x=I$ by putting all of
	the given formulas \textit{literally} in prover9, see
	\texttt{groups\_a1.in}.

	Using mace4, we show that, in general, $inv(x*y)=inv(x)*inv(y)$ does not
	hold. The smallest group for which this does not hold has size $6$, see
	\texttt{groups\_a2.in} and appendix \ref{apx: groups}.
\item
	Using mace4, we show that the smallest non-Abelian group has size $6$,
	again by \textit{literally} putting in the formulas provided, see
	\texttt{groups\_b.in} and appendix \ref{apx: groups}.
\item
	We use the fact that $x^2 = x * x$, $x^3 = x * (x * x)$ and $x^4 = x * (x *
	(x * x))$ this makes encoding $x^n = I$ straightforward for $n = 2, 3, 4$.

	Using prover9, we show that for $n=2$ all such groups are Abelian, see \texttt{groups\_c1.in}.

	Using mace4, we show that for $n=3$ the group is not Abelian in general and
	the smallest counterexample has size $27$, see \texttt{groups\_c2.in}.

	Using mace4, we show that for $n=4$ the group is not Abelian and the
	smallest counterexample has size $8$, see \texttt{groups\_c3.in} and
	appendix \ref{apx: groups}.
\end{enumerate}
% section groups (end)

\appendix
\pagebreak
\section{Groups}
\label{apx: groups}
\begin{enumerate}[label=\alph*)]
	\item
		Smallest finite group for which $inv(x*y)=inv(x)*inv(y)$ does not hold.
		\begin{verbatim}
		interpretation( 6, [number=1, seconds=0], [

		    function(I, [ 0 ]),

		    function(c1, [ 1 ]),

		    function(c2, [ 2 ]),

		    function(inv(_), [ 0, 1, 2, 4, 3, 5 ]),

		    function(*(_,_), [
		        0, 1, 2, 3, 4, 5,
		        1, 0, 3, 2, 5, 4,
		        2, 4, 0, 5, 1, 3,
		        3, 5, 1, 4, 0, 2,
		        4, 2, 5, 0, 3, 1,
		        5, 3, 4, 1, 2, 0 ])
		]).
		\end{verbatim}

	\item
		Smallest non-abelian group.
		\begin{verbatim}
		interpretation( 6, [number=1, seconds=0], [

			    function(I, [ 0 ]),

			    function(c1, [ 1 ]),

			    function(c2, [ 2 ]),

			    function(inv(_), [ 0, 1, 2, 4, 3, 5 ]),

			    function(*(_,_), [
			        0, 1, 2, 3, 4, 5,
			        1, 0, 3, 2, 5, 4,
			        2, 4, 0, 5, 1, 3,
			        3, 5, 1, 4, 0, 2,
			        4, 2, 5, 0, 3, 1,
			        5, 3, 4, 1, 2, 0 ])
		]).
		\end{verbatim}

	\item
		\pagebreak
		Smallest non-abelian group with $x^4 = I$.
		\begin{verbatim}
		interpretation( 6, [number=1, seconds=0], [

			    function(I, [ 0 ]),

			    function(c1, [ 1 ]),

			    function(c2, [ 2 ]),

			    function(inv(_), [ 0, 1, 2, 4, 3, 5, 6, 7 ]),

			    function(*(_,_), [
			        0, 1, 2, 3, 4, 5, 6, 7,
			        1, 0, 3, 2, 5, 4, 7, 6,
			        2, 4, 0, 6, 1, 7, 3, 5,
			        3, 5, 1, 7, 0, 6, 2, 4,
			        4, 2, 6, 0, 7, 1, 5, 3,
			        5, 3, 7, 1, 6, 0, 4, 2,
			        6, 7, 4, 5, 2, 3, 0, 1,
			        7, 6, 5, 4, 3, 2, 1, 0 ])
		]).
		\end{verbatim}
\end{enumerate}

\end{document}
