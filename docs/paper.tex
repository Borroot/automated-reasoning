\documentclass[a4paper]{article}

\usepackage{amsfonts}

\author{Jelmer Firet, Bram Pulles}
\title{\textbf{Automated Reasoning\\Practical Assignment -- Part 1}}

\begin{document}
\maketitle

\section{Introduction}
We have chosen to use Z3 for solving all of the assignments. In particular we
use the pythonic Z3 library.

\section{Pallets}
In this task we need to distribute pallets of various goods over a limited
number of trucks. To this end we have formulated the problem as a system of
linear equations. We created an integer variable $v_{g,t}$ for every $good$ and
$truck$ combination such that $v_{g,t} \in good \times truck$. This variable
describes how many pallets of a certain good are in the specific truck. Using
just these variables we can describe all of the constraints.

\begin{itemize}
	\item
		We can only have a positive number of pallets in a truck.
		\[ \forall_{v_{g,t} \in good \times truck}\ v_{g,t} \geq 0 \]
	\item
		Every truck has at most eight pallets.
		\[ \forall_{t \in truck}\ \left( \sum_{g \in good} v_{g,t} \right) \leq 8 \]
	\item
		Every truck can carry at most 8000 kg. Let $w$ be a function giving the
		weight of a given good $w: good \to \mathbb{N}$.
		\[ \forall_{t \in truck}\ \left( \sum_{g \in good} w(g) \right) \leq 8000 \]
	\item
		For every good all the pallets are distributed. Except for prittles, as
		no number of pallets is specified, we need to maximize this. Let $p$ be
		a function giving the number of pallets of a given good $p: good \to
		\mathbb{N}$.
		\[
		\forall_{g \in good \setminus prittles}\
		\left( \sum_{t \in truck} v_{g,t} \right) = p(g)
		\]
	\item
		Only three trucks can contain skipples. Let $t_i$ be the $i-$th truck,
		starting at $i = 1$.
		\[ \forall_{t_i \in truck, i > 3 }\ v_{skipples,t} = 0 \]
	\item
		No two pallets of nuzzles may be in the same truck.
		\[ \forall_{t \in truck}\ v_{nuzzles,t} \leq 1 \]
	\item
		Prittles and crottles are not allowed to be put in the same truck.
		\[ \forall_{t \in truck}\ v_{prittles,t} = 0 \lor v_{crottles,t} = 0 \]
\end{itemize}

All of the constraints described above can be easily converted to Z3. In order
to let Z3 automatically maximize the number of pallets with prittles we set the
maximisation function to maximize the formula shown below.

\[ \sum_{t \in truck} v_{prittles,t} \]

Running our program \textit{without} the last constraint (question 1) gives us
the result shown below. Every row is a truck, every column is the good starting
with that letter. As can be seen from the table, all the pallets can be
distributed and there are a total of $8 + 5 + 5 + 4 = 22$ prittles pallets.
\begin{verbatim}
    n s c d p
0 : 0 0 0 0 8
1 : 0 8 0 0 0
2 : 1 0 2 0 5
3 : 1 0 2 0 5
4 : 0 0 2 6 0
5 : 0 0 0 4 4
6 : 1 0 2 5 0
7 : 1 0 2 5 0
\end{verbatim}

Running our program \textit{with} the last constraint (question 2) gives us the
result shown below. Again, all the pallets can be distributed and there are a
total of $4 + 8 + 8 = 20$ prittles pallets.

\begin{verbatim}
    n s c d p
0 : 0 4 0 0 4
1 : 1 2 2 1 0
2 : 0 2 2 4 0
3 : 0 0 0 0 8
4 : 1 0 2 5 0
5 : 1 0 2 5 0
6 : 1 0 2 5 0
7 : 0 0 0 0 8
\end{verbatim}

Our program and formalisation are generalised under the number of trucks, the
truck maximum weight and maximum number of pallets. It is also very easy to add
more goods, change their weight or change the number of pallets.

\section{Chip design}


\section{Dinner}


\section{Program safety}


\end{document}
